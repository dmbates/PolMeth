\documentclass[pdf]{beamer}
\mode<presentation>{\usetheme{Frankfurt}\usecolortheme{seahorse}}
\title{Programming Languages for Data Analysis}
\date{July 14, 2017}
\author{Douglas Bates\\Dept. of Statistics\\U. of Wisconsin-Madison}
\begin{document}
  \begin{frame}
    \maketitle
  \end{frame}

  \section{Preliminary Julia info}

  \begin{frame}\frametitle{For the impatient}
    \begin{itemize}
      \pause
      \item The final part of this talk will be about the
      \href{https://julialang.org}{Julia} programming language
      \pause
      \item For an overview of \href{https:julialang.org}{Julia} for econometrics,
      see \href{https://quantecon.org}{quantecon.org}
      \pause
      \item The \href{https://lectures.quantecon.org/}{lectures} tab provides a
      \href{https://python.org}{Python} version and a
      \href{https://julialang.org}{Julia} version
      \pause
      \item The \href{https://lectures.quantecon.org/about_lectures.html}{About}
      tab is a very good discussion of why these two languages were
      chosen.
    \end{itemize}
  \end{frame}

  \begin{frame}\frametitle{More Julia resources}
    \begin{itemize}
      \pause
      \item The original developers formed a company,
      \href{https://juliacomputing.com}{Julia Computing}, to provide
      consulting services and oversee the development of the language.
      \pause
      \item The \href{https://juliacomputing.com/case-studies/}{case studies}
      section is of particular interest
      \pause
      \item Registered Julia packages are at
      \href{https://pkg.julialang.org}{pkg.julialang.org}
      \pause
      \item Note that Julia packages are
      \href{https://en.wikipedia.org/wiki/Git}{git} repositories, usually
      housed on \href{http://github.com}{github.com}
      \pause
      \item See also
      \href{https://pkg.julialang.org/pulse.html}{Julia Package Ecosystem Pulse}
      \pause
      \item Documentation for the language itself is at
      \href{https://docs.julialang.org}{docs.julialang.org}
    \end{itemize}
  \end{frame}

  \section{Background on S and R}
  \begin{frame}\frametitle{Language shapes the way we think - Benjamin Lee Whorf}
    \begin{itemize}
      \pause
      \item I have been doing data analysis programming for a long time
      \begin{itemize}
        \pause
        \item took my only computing course in 1967 - didn't like it
        \pause
        \item in the 70's wrote Fortran code that I hope no one ever discovers
        \pause
        \item late 70's read Kernighan and his co-authors, saw programming
        differently
        \pause
        \item was able to use sophisticated languages, almost by accident, in
        thesis research
        \pause
        \item some SPSS and SAS use, Minitab for teaching
        \pause
        \item heard about the work on S by John Chambers and co. at Bell Labs
      \end{itemize}
    \end{itemize}
  \end{frame}

  \begin{frame}\frametitle{What was different about S?}
    \begin{itemize}
      \pause
      \item
      an interactive language (REPL - read-eval-print-loop)
      \pause
      \item functional language (in the sense of defining and calling functions)
      \pause
      \item
      heterogeneous, self-describing, recursive, extensible data structures
      \begin{itemize}
        \pause
        \item
        contrast with flat-files structures in SPSS, SAS; vectors (columns)
        in Minitab
      \end{itemize}
      \pause
      \item
      random-access memory based, not a filter
      \begin{itemize}
        \pause
        \item required a ``second megabyte of memory''
      \end{itemize}
      \pause
      \item explicit interfaces to compiled code
        \begin{itemize}
        \pause
        \item
          original versions were more of a wrapper around numerical and
          graphics code bases
        \end{itemize}
    \end{itemize}
  \end{frame}

  \begin{frame}\frametitle{What was the same with S?}
    \begin{itemize}
      \pause
      \item explicitly designed for data analysis and graphics
      \pause
      \item provision for missing data was built in at a very low level
      \pause
      \item
      internally, data structures were always vectors, possibly with ``attributes''
      \pause
      \item ``lists'' are actually vectors of pointers, not linked lists as in
      Lisp
      \pause
      \item ``semi-proprietary'' software
      \begin{itemize}
        \pause
        \item AT\&T couldn't market S (or the Unix operating system)
        \item some universities became ``beta-test sites''
        \item U. of Washington Stats. Dept. spun off ``StatSci'' and marketed S-PLUS
      \end{itemize}
    \end{itemize}
  \end{frame}

  \begin{frame}\frametitle{The 90's and the birth of R}
        \begin{itemize}
    \pause
    \item
      Open-Source software gained traction (perl, python, emacs, GNU/Linux, MySQL,
      PostgreSQL) in some areas
      \begin{itemize}
      \pause
      \item Servers were often characterized as LAMP (Linux, Apache, MySQL,
        Perl/Python)
      \item First releases of both Python and Linux were in 1991
      \end{itemize}
    \pause
    \item
      mid-90's Ross Ihaka and Robert Gentleman started work on a language
      ``not-unlike S''
      \begin{itemize}
        \pause
        \item S-PLUS had been ported to DOS/Windows but not to Macintosh
        \item essentially a ``clean room'' reimplementation of the S language
        \item Martin M\"{a}chler contributed so many patches they gave him an account
        \item Martin encouraged release under the GPL
        \item others joined the party, 1997 R-Core was formed.
      \end{itemize}
    \end{itemize}
  \end{frame}

  \begin{frame}\frametitle{The 90's and CRAN}
    \begin{itemize}
      \pause
      \item Kurt Hornik and Fritz Leisch, then at TU-Wien, created the
      ``Comprehensive R Archive Network'' (\textbf{CRAN})
      \begin{itemize}
        \pause
        \item patterned after CTAN (TeX) and CPAN (Perl) archive networks
      \end{itemize}
      \pause
      \item
      Encouraged users to become developers and researchers to provide a
      reference implementation of their methods
      \pause
      \item Promoted the development of package standards, testing etc.
      \pause
      \item One important facility added later (Uwe Ligges) was ``Win-builder'' to
      create binary Windows packages
      \begin{itemize}
        \pause
        \item most developers were on Linux, most users on Windows
      \end{itemize}
    \end{itemize}
  \end{frame}

  \begin{frame}\frametitle{The aughts - R fourishes}
    \begin{itemize}
      \pause
      \item
      Initially R was considered by many to be a ``poor man's S-PLUS''
      \pause
      \begin{itemize}
        \item that is ``Freeware'' couldn't possibly be as good as commercial software
      \end{itemize}
      \pause
      \item
      Eventually the open-source model was recognized as producing high-quality code
      \begin{itemize}
        \pause
        \item Eric S. Raymond, ``Given enough eyeballs, all bugs are shallow.''
      \end{itemize}
      \pause
      \item
      R-1.0.0 was released on February 29, 2000 - the day that didn't exist
      \pause
      \item
      CRAN went from 10's to 100's to 1000's of packages
      \pause
      \item
      useR! conferences started, \emph{R Newsletter} later to become \emph{R
      Journal} founded
      \pause
      \item papers, books, online resources, became available
    \end{itemize}
  \end{frame}

  \begin{frame}\frametitle{The aughts - other Open-Source languages enter the mix}
    \begin{itemize}
      \pause
      \item S and R were designed for data analysis and graphics, not general purpose programming
      \begin{itemize}
        \pause
        \item
        as Ross said, ``we thought maybe a couple of hundred people total  would use it''
      \end{itemize}
      \pause
      \item every language is a trade-off between pre-defined, high-level operations and low-level building blocks
      \begin{itemize}
        \pause
        \item
        R is oriented to high-level tools: vectors, matrices, data frames are okay; scalars, loops, low-level logic not so much
        \item
        the number of internal data types is surprisingly small (32-bit integers, 64-bit floats and character strings)
      \end{itemize}
      \pause
      \item
      languages like Python provided more flexibility but without the data
      science specific structures
      \begin{itemize}
        \pause
        \item required add-ons like numpy, pandas, etc.
      \end{itemize}
    \end{itemize}
  \end{frame}

  \begin{frame}\frametitle{The teens and enhancements like RStudio and Rcpp}
  \begin{itemize}
    \pause
    \item RStudio - definitive IDE for R
    \pause
    \begin{itemize}
      \item Introduced in 2011
      \item Freely available for personal use - important for teaching
      \item J.J. Allaire brought Hadley Wickham on as Chief Scientist.  More hires followed.
      \item RStudio is now much more than an IDE - the most creative group in the R community
      \item Fast moving group in contrast to more conservative R core
    \end{itemize}
    \item Rcpp - ``seamless'' integration of R and C++
    \pause
    \begin{itemize}
      \item Integrate ease of use of R with speed of a compiled language relatively painlessly
      \item But - C++
      \item There is a fundamental mismatch in the languages (static vs dynamic)
    \end{itemize}
  \end{itemize}

  \end{frame}

\section{Hitting the wall with R}

  \begin{frame}\frametitle{My adventures fitting mixed-effects models - nlme}
    \begin{itemize}
      \pause
      \item In the mid-90's I was dragged into research on models with fixed- and random-effects, called \textit{mixed-effects models} in Statistics
      \pause
      \item Jos\'e Pinheiro and I wrote a package for S called \textit{nlme} to fit linear and nonlinear mixed-effects models
      \begin{itemize}
        \pause
        \item One of the first packages ported to R.
        \item Also a Springer book published in 2000.
        \item About 14,000 lines of R code and 3,000 lines of C code
        \item Used S3 classes and methods.
        \item Compiled code called through \texttt{.C} and \texttt{.Call}
      \end{itemize}
      \pause
      \item As soon as it was released I started thinking there were better ways to do this
    \end{itemize}
  \end{frame}

  \begin{frame}\frametitle{My adventures fitting mixed-effects models - lme4}
    \begin{itemize}
      \pause
      \item In the aughts Martin M\"achler and I started work on mixed-effects models using S4 classes and methods
      \begin{itemize}
        \pause
        \item Incorporated sparse matrix methods for which we needed to access more linear algebra software
        \item Initially the linear algebra was part of \textbf{lme4}, later split to the \textbf{Matrix} package
        \item Later \textbf{Rcpp}'ized \textbf{Eigen} (C++ matrix package) as \textbf{RcppEigen}
        \item Needed better optimization software for difficult case
        \item Ended up using S3, S4, reference classes, $\dots$
        \item In the teens, Ben Bolker became the primary maintainer.
        \item Now about 6,000 lines of R code, 2700 lines of C/C++ code
        \item Fits both linear mixed models and generalized linear mixed models
        \item Faster and more reliable than \texttt{nlme} although coverage of two packages not identical
        \item Can fit models with crossed random effects, such as for Subject/Item data
      \end{itemize}
    \end{itemize}
  \end{frame}

  \begin{frame}\frametitle{Why not stop at lme4?}
    \begin{itemize}
      \pause
      \item Code is unwieldy and opaque - difficult to maintain and/or extend
      \pause
      \item Still problems with
      \pause
      \begin{itemize}
        \item optimizers
        \item linear algebra
        \item using large objects in iterative algorithms
        \item the static/dynamic barrier in general
      \end{itemize}
      \pause
      \item Many of these issues are the result of S/R initial design decisions
      \begin{itemize}
        \item Lazy evaluation
        \item Functional semantics
      \end{itemize}
    \end{itemize}
  \end{frame}

  \section{Enter Julia}

  \begin{frame}\frametitle{The Julia language}
    \begin{itemize}
      \pause
      \item Julia was developed by very skilled programmers explicitly for quantitative work
      \pause
      \item First public release in 2011 so relatively immature.  Still working on important infrastructure (e.g. DataFrames)
      \pause
      \item Developers has extensive experience with many languages (Python, R, Matlab, ...), but wanted more
      \item Functions/methods, classes (a.k.a, types) as in R but much more rigorous
      \item Flexibility and speed can be achieved with Just-In-Time compilation (JIT)
    \end{itemize}
  \end{frame}

  \begin{frame}\frametitle{MixedModels package}
  \begin{itemize}
    \item Similar in coverage to the lme4 package
    \item About 2700 lines of Julia code
    \item Algorithmic enhancements relative to lme4
    \item Usually about 10x faster than lme4
  \end{itemize}

  \end{frame}
\end{document}
